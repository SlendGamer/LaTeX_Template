\chapter{Hauptteil}
\label{cha:hauptteil}
Anforderungsdefinition, Anforderungsanalyse, Lösungsgenerierung, Lösungsbewertung, Umsetzung \\
ggf. in mehreren sinnvollen Gliederungspunkten

Der Text soll knapp und klar sein und die wesentlichen Gedanken der Arbeit beinhalten. Ein gewähltes Verfahren oder ein bestimmter Lösungsweg muss begründet werden. Es ist nicht notwendig, alle Vorversuche einzeln zu schildern. Bei Versuchen sind Voraussetzungen und Vernachlässigungen sowie die Anordnung, Leistungsfähigkeit und Messgenauigkeit der Versuchsanordnung anzugeben.

Die Ergebnisse der Arbeit sind unter Berücksichtigung der Voraussetzungen ausführlich zu diskutieren und
mit den bereits bekannten Anschauungen und Erfahrungen zu vergleichen.

Ziel der Arbeit ist es, eindeutige Folgerungen und Richtlinien für die Praxis zu finden.

\begin{itemize}
	\item Anforderungsdefinition
	\item Anforderungsanalyse
	\item Lösungsgenerierung
	\item Lösungsbewertung
	\item Umsetzung
\end{itemize}

\section{Schwerpunkt 1}

\subsection{Unterpunkt 1}

\subsection{Unterpunkt 2}