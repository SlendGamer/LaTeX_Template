\chapter{Grundlagen}
\label{cha:Grundlagen}

\section{Schalltheorie} 
\label{sec:Schalltheorie}
Zur Planung von Beschallungsanlagen sind Kenntnisse zur Schallaufnahme, -bearbeitung und -verteilung \textbf{zwingend notwendig}. Schall ist eine zeitlich periodische Bewegung von Luftmolekülen, die sich über benachbarte Moleküle als Folge der Druckschwankungen in alle Raumrichtungen ausbreitet. Somit ist Schall eine Form der Energie, die sich durch Druckwellen in der Luft ausbreitet. \\
Es wird zwischen folgenden Formen der Wellenausbreitung unterschieden:
\begin{compactenum}
\item kugelförmig, von einer punktförmigen Quelle ausgehend
\item zylindrisch, von einer stabförmigen Quelle ausgehend 
\item eben, von einer planen Quelle ausgehend
\end{compactenum}
Im Rahmen der Projektarbeit werden Sprach-, Alarm- und Hinweissignale betrachtet, wobei Lautsprecher als Schallquellen dienen. Bei der Umwandlung jeglicher Schallsignale in elektrische Signale kommt es unvermeidlich zu Einschränkungen in der Dynamik, Frequenzcharakteristik und Verständlichkeit.
Dynamik meint hierbei den Unterschied zwischen dem leisesten und lautestem Geräusch bzw. den Unterschied zwischen dem Spitzenwert und dem minimalen Wert des umgesetzten Signals. Verständlichkeit bezieht sich auf die übertragenen Sprachsignale und bedeutet die möglichst semantisch fehlerfreie Aufnahme einer gesprochenen Nachricht für die Zuhörer über die Lautsprecher. Die Frequenzcharakteristik bzw. das Frequenzspektrum betrachtet die Zusammensetzung eines Schallsignals aus den verschiedenen Frequenzanteilen.

\subsection{Sprache} 
\label{sub:Sprache} 
Sprache ist ein Konstrukt aus Sätzen, Wörtern und Buchstaben, die weiter in Konsonanten und Vokale unterteilt werden können. Aus der schwankenden Lautstärke und Tonhöhe ergibt sich weiterhin ein charakteristisches, von der Person und Sprechweise abhängiges Frequenzspektrum. Während Vokale, die für den Lautstärkeeindruck verantwortlich sind, im Frequenzspektrum unterhalb von 1000 Hz liegen, befinden sich Konsonanten oberhalb der 1000 Hz und dienen zur Erkennbarkeit der Wörter. 
Die Prämisse bei der Schallübertragung ist dabei, das komplette Sprachspektrum so unverändert wie möglich an die Zuhörer weiterzugeben. Es gibt eine große Zahl akustischer Erscheinungen, die Veränderungen an diesem Sprachspektrum hervorrufen, welche die Verständlichkeit für die Zuhörer verschlechtern. Aus diesem Grund werden verschiedene Techniken verwendet, um diesen Einflüssen entgegenzuwirken.

\subsection{Dezibel} 
\label{sub:Dezibel}
Das Dezibel ist in der Ton- und Kommunikationstechnik eine sehr gebräuchliche Einheit zur Beschreibung von Pegeln, Verstärkungen und Dämpfungen. Zunächst einmal muss erwähnt werden, dass Dezibel eine \textit{Pseudoeinheit} ist. D.h. sie ist keine SI-Einheit, sondern eine Verhältniseinheit, mit der das Verhältnis zweier Zahlen beschrieben werden kann. Diese Notation macht es möglich, Skalenwerte sinnvoll und nach Bedarf zu komprimieren oder zu spreizen und erlaubt weiterhin eine deutliche Vereinfachung von Berechnungen mit großen Zahlenwerten, da sie mit dem Dekadischen Logarithmus berechnet wird.  Die menschlichen Sinne arbeiten ebenfalls mit logarithmischen Maßstäben. Dies bedeutet, dass bei jedem Reiz die kleinste wahrnehmbare Veränderung proportional zum Niveau des bereits vorhandenen Reizes ist. Die Einheit \textit{Bel} ist als ${\log_{10}\,P1/P2}$ definiert und somit ist ein \textit{Dezibel} gleich ${10\cdot\log_{10}\,P1/P2}$. Es wird deutlich, dass eine Erhöhung um rund 3 dB einer Verdopplung der Leistung und eine Zunahme auf das 100-fache einem Wert von 20 dB entspricht.
Obwohl das Dezibel nur ein Verhältnismaß ist, kann es, wenn ein Referenzwert angegeben wird, auch für die Angabe absoluter Werte eingesetzt werden, was im nächsten Abschnitt \ref{sub:Lautstaerke} dargestellt ist.

\subsection{Lautstärke} 
\label{sub:Lautstaerke}
Lautstärke ist die subjektive Wahrnehmung des Schalldruckpegels. Wenn hingegen Schallsignale gemessen werden, bezieht man sich auf Veränderungen des Luftdrucks und arbeitet deshalb mit dem Verhältnis zum Schalldruck, dem Schalldruckpegel (engl. SPL). Der angesetzte Referenzwert ist dabei der Schalldruck, den ein durchschnittlicher menschlicher Zuhörer bei einer Frequenz von 1 kHz noch wahrnehmen kann, die sog. \textit{Hörgrenze}. Sie beträgt bei genannter Frequenz ${20 mN/m^2 = 20 mPa = 2\cdot 10^{-5}\,Pa}$

Wenn der Schalldruckpegel zunimmt, unmittelbar bevor der Schall als schmerzhaft wahrgenommen wird, ist ein Punkt erreicht, der als \textit{Schmerzgrenze} bezeichnet wird. Bei der Referenzfrequenz von 1 kHz beträgt dieser Wert 20 Pascal. Wenn der absolute Referenzwert der Hörgrenze als 0 dB (SPL) angenommen wird, liegt der Wert der Schmerzgrenze bei:
\begin{center}
$20\cdot\log\frac{20\,Pa}{2\cdot 10^{-5}\,Pa} = 120\,dB\,(SPL)$
\end{center}
Die folgende Grafik zeigt den Schalldruckpegel in dB (SPL) für verschiedene Umgebungen und Geräuschquellen:

\begin{figure}[H]
\centering
\includegraphics[width=1\linewidth]{images/Laermpegel}
\caption{Lärmpegel in unterschiedlichen Umgebungen}
\label{fig:Laermpegel}
\end{figure}

Wenn zwei nicht korrelierte Schallquellen kombiniert werden, darf nur ihre Schallintensität (Energie) addiert werden:
\begin{center}
$L_s = 10\cdot\log[10^{\frac{L_1}{10}}+10^{\frac{L_2}{10}}]$
\end{center}
Zwei verschiedene Schallquellen, die beide 90 dB (SPL) leisten, werden somit wahrgenommen als:
\begin{center}
$L_s = 10\cdot\log[10^9+10^9] = 93\,dB$
\end{center}

\subsection{Hörcharakteristik} 
\label{sub:Hoercharakteristik}
Die Hörgrenze ist die Untergrenze einer Reihe von Kurven des gleichen Lautstärkeempfindens, wohingegen die Schmerzgrenze die Obergrenze darstellt. Das Lautstärkeempfinden hängt dabei stark von der Frequenz des Tones ab, weshalb für einen mit 30 Hz schwingenden Ton ein Schalldruckpegel von 60 dB erforderlich ist, um den gleichen Lautstärkeeindruck wie bei 0 dB und 1 kHz zu erzeugen.

\begin{figure}[H]
\centering
\includegraphics[width=0.5\linewidth]{images/Akustik_db2phon}
\caption{Kurven gleicher Lautstärkepegel}
\label{fig:Lautstaerkewahrnehmung}
\end{figure}

Das bedeutet zum einen, für die Erzeugung tiefer Töne einer gegebenen Lautstärke ist viel mehr Energie erforderlich, als für einen Ton von 2 oder 3 kHz. Zum anderen erhält der Zuhörer bei der Wiedergabe von weißen Rauschen mit einem Pegel von 20 dB einen Lautstärkeeindruck, der der 20-phon-Kurve entspricht. Wird der Rauschpegel jedoch erhöht, erzeugt dasselbe Rauschen einen anderen Eindruck, da das Signal abhängig vom Wiedergabepegel unterschiedlich wahrgenommen wird.

Ein Schalldruckmesser arbeitet aus diesem Grund mit verschiedenen Filterkurven, die das subjektive Verhalten des Gehörs abbilden. International sind drei Kurventypen definiert, die als A-, B-, und C-Gewichtungskurve bezeichnet werden. 

Die A-Kurve sollte nur bei Messungen unter 40 dB eingesetzt werden, jedoch verfügen viele einfache Schalldruckpegelmesser lediglich über einen Filter der A-Kurve, weshalb die Mehrzahl der Messungen in der Akustik mit dieser Gewichtung vorgenommen werden. Die Ergebnisse werden in dBA (SPL) angegeben.

\section{Verstärkung und Signalbearbeitung} 
\label{sub:Verstaerkung}
Mittelpunkt von vielen Beschallungsanlagen ist ein Mischpult. Es ersetzt die vielen Vorverstärker und ist gleichzeitig die Regeleinrichtung, an der alle Mikrofone, Wiedergabegeräte usw. zusammenlaufen. Das Mischpult übernimmt diese verschiedenen Eingangssignale und fügt sie zu einem ausgewogenen Summensignal zusammen, welches anschließend den Eingängen der Leistungsverstärker bzw. den Monitor-Lautsprechern zugeführt werden.

\subsection{Vorverstärker}
\label{sub:Vorverstaerker}
Auch wenn sie sich häufig im selben Gehäuse befinden, bestehen Verstärker für Beschallungsanlagen eigentlich aus einem Vorverstärker und dem End- oder Leistungsverstärker. Falls erforderlich, können am Ausgang eines Vorverstärkers mehrere Leistungsverstärker parallel angeschlossen werden, um eine größere Anzahl von Lautsprechern zu versorgen.
Der Vorverstärker dient dazu, die Ausgangssignale der Anschlussgeräte auszuwählen bzw. zu mischen und auf einen bestimmten Pegel anzuheben, der zur Ansteuerung der Leistungsverstärker ausreicht. Er enthält auch die Klangregelung, die Eingangsempfindlichkeits-Einstellung und den Summen-Lautstärkeregler.

Der Vorverstärker dient dazu, die niedrigen Signalspannungen der Quellen auf einen Pegel, meistens 500 mV bis 1 V, anzuheben, der zur Ansteuerung der Leistungsvertärker ausreicht. Ein Vorverstärker kann die folgenden Eingänge haben:
\begin{compactenum} 
\item dynamisches Mikrofon: 0,25 mV
\item BPE- oder Elektretmikrofon: 1 mV
\item Kondensatormikrofon: 3 mV
\item dynamischer Tonabnehmer: 5 mV
\end{compactenum}

Diese Spannungspegel stellen dabei jeweils die Empfindlichkeit des Eingangs dar, d.h. in welchem Pegel-Bereich sich das eingehende Spannungssignal befinden muss, damit eine sinnvolle Verstärkung erreicht wird. Dieser Pegel darf jedoch auch nicht grob überschritten werden, sonst kommt es zum sog. \textit{Klirren bzw. Clipping}, wodurch das Signal verzerrt und unverständlich wird (siehe Abschnitt \ref{sub:Technische Daten1}). Bei weiterer Überschreitung wird der jeweilige Vorverstärkerkreis womöglich zerstört, wobei viele Verstärker aber schon entsprechende Schutzschaltungen beinhalten.

Am häufigsten werden die Empfindlichkeitsbereiche \textbf{0,5 bis 1,5 mV} für Mikrofoneingänge und \textbf{100 mV bis 1 V} für den Anschluss von sonstigen Abspielgeräten benötigt.

\subsection{Klangregelung}
\label{sub:Klangregelung}
Die Klangregelung ändert den Frequenzgang des Verstärkers im Bezug auf das eingehende Signal nach einstellbaren Parametern am Verstärker. Diese Parameter sind normalerweise die Bass- und Höheneinstellung. Die Schaltungen, aus denen diese aufgebaut sind, bestehen aus Verstärker- bzw. Dämpfungsschaltungen, die in bestimmten Frequenzbändern folgendermaßen arbeiten:
\begin{compactenum}
\item  Wenn der Höhen- oder Bassregler aus der Mittenposition nach rechts gedreht wird, nimmt die Verstärkung zu, und die Frequenzen im Arbeitsbereich der Schaltung werden verstärkt, was zu einer Erhöhung der Lautstärke im entsprechenden Frequenzbereich führt. Eine Anhebung der Höhen ist besonders dann sinnvoll, wenn auch bei starkem Umgebungsgeräusch eine gut verständliche Sprachübertragung erreicht werden soll (siehe Abschnitt \ref{sub:Sprache}).
\item Wenn das Potentiometer aus der Mittelstellung in die entgegengesetzte Richtung gedreht wird, werden die Höhen bzw. Bässe entsprechend gedämpft. Ein Abregeln der Bässe ist dann vorteilhaft, wenn lange Nachhallzeiten bei niedrigen Frequenzen zu Problemen führen (siehe Abschnitt \ref{sub:in Innenraeumen}).
\item Eine Anhebung der Bässe bzw. Absenkung der Höhen ist selten erforderlich. In sehr stark gedämpften Räumen kann es nötig werden, die Bässe anzuheben, um eine ausgewogene Schallwiedergabe zu erreichen. Dabei muss jedoch darauf geachtet werden, die Lautsprecher nicht zu überlasten.
\end{compactenum}
Entzerrer bieten im Gegensatz dazu umfangreiche Einstellmöglichkeiten im gesamten Frequenzspektrum. Sie können dazu verwendet werden, den Frequenzgang von Lautsprechern zu optimieren und darüber hinaus zur Entzerrung des gesamten Signalwegs eingesetzt werden.

Man unterscheidet zwischen folgenden Entzerrern:
\begin{compactenum}
\item \textbf{Hoch- und Tiefpassfilter} sind dazu vorgesehen, den Frequenzbereich zu beschneiden, d.h. alle Signale unter- bzw. oberhalb eines bestimmten Frequenzbandes sehr stark zu bedämpfen.
\item \textbf{Parametrische Entzerrer} sind Einheiten mit 3 oder 4 Filtern und der Möglichkeit, die Arbeitsfrequenz einzustellen. Sie verfügen über einen Verstärkungsregler und eine Einstellungsmöglichkeit für die Bandbreite. Falls gefordert, lässt sich damit ein sehr schmales Frequenzband bearbeiten, ohne die benachbarten Frequenzen zu beeinflussen.
\item \textbf{Grafische Equalizer} sind Entzerrer mit fest vorgeschriebenen Frequenzen und meistens bis zu 30 regelbaren Frequenzbändern in Form von Schiebereglern. Jeder Regler regelt ein sehr schmales Frequenzband, dessen Wirkung jeweils in der Mitte des Bandes am stärksten ist.
\end{compactenum}

Die theoretisch mögliche Erhöhung der Lautstärke durch Einsatz einer Verstärkeranlage wird häufig durch akustische Rückkopplung eingeschränkt. Die Rückkopplung bzw. Mitkopplung wird dadurch verursacht, dass verstärkter Schall von den Lautsprechern wieder zum Mikrofon gelangt. Bei kontinuierlicher Erhöhung der Verstärkerleistung, wird ein Punkt erreicht, bei dem die Anlage spontan anfängt zu schwingen. Der Einsatz des Entzerrers zur Vermeidung solcher Rückkopplungen erhöht gleichzeitig die Klangqualität und die Gesamtverstärkung der Anlage.

\subsection{Leistungsverstärker}
\label{sub:Leistungsverstaerker}
Der Leistungsverstärker, der häufig über ein eigenes Gehäuse verfügt, dient dazu, die Ausgangsspannung des Vorverstärkers oder Mischpults auf einen Leistungspegel anzuheben, mit dem die Lautsprecher versorgt werden können. Je nach Konzeption des Herstellers kann die nötige Eingangsspannung zur Vollaussteuerung des Leistungsverstärkers zwischen 100 mV und 10 V liegen.
Verstärker für den Einsatz in Beschallungsanlagen arbeiten in der sog. \textit{100-Volt-Technik}. Diese Technik hat den Vorteil, dass die Entfernung zu den Lautsprechern sehr groß sein darf. Sie wird im folgenden Abschnitt \ref{sub:100-Volt-Technik} näher behandelt.

\subsection{100-Volt-Technik}
\label{sub:100-Volt-Technik}
Verstärker für einfache Lautsprecheranlagen haben normalerweise einen niederohmigen Ausgang mit einer Impedanz von 2, 4 oder 8 Ohm. Dabei ist zu beachten, dass die Impedanz der angeschlossenen Lautsprecher der Ausgangsimpedanz des Verstärkers entsprechen soll und dass die Belastbarkeit der Lautsprecher grundsätzlich höher sein soll, als die Ausgangsleistung des Verstärkers, um eine Überlastung der Lautsprecher zu vermeiden.
Die Lautsprecher können in einer Serien- bzw. Parallelschaltung angeschlossen werden, um eine genaue Anpassung an die niedrige Ausgangsimpedanz des Verstärkers zu erreichen. Dies bedeutet jedoch, dass die Anpassung an den Leistungsverstärker sehr kompliziert wird. In solchen Fällen und allen Anwendungen mit langen Lautsprecherleitungen ist die 100-Volt-Technik einzusetzen. Verstärker mit niederohmigen Ausgängen kommen in diesen Anwendungen eher selten vor.

Diese Technik arbeitet mit Übertragern, die in den Leistungsverstärkern eingebaut sind und über Abgriffe verfügen, an denen Ausgangsspannungen von 100, 70 und 50 Volt abgenommen werden können. Übertrager in den Lautsprechern sorgen für die Anpassung an die niedrige Impedanz des Lautsprechers.

Die 100-Volt-Technik ermöglicht ein hohes Maß an Flexibilität bei der Planung und im Betrieb von Beschallungsanlagen.
\begin{compactenum}
\item Durch das Hochtransformieren der Ausgangsspannung des Verstärkers wird die Stromstärke bei gegebener Leistung erheblich verringert, was Leistungsverluste vermindert und auch bei hoher Leistung den Einsatz von großen Kabelquerschnitten erspart.
\item Wegen der geringen Leistungsverluste sind große Leitungslängen möglich.
\item Aufgrund der Impedanzanpassung, können alle Lautsprecher parallel geschaltet werden.
\end{compactenum}

Die 100-Volt-Technik ist vergleichbar mit der Stromversorgung in einem normalen Haushalt. Solange die gesamte Leistungsaufnahme der angeschlossenen Lautsprecher nicht größer ist, als die Nennleistung des Verstärkers, ist es völlig unerheblich, ob 1 oder 150 Lautsprecher angeschlossen sind. Wenn Lautsprecher am 100-Volt-Abgriff des Verstärkers angeschlossen werden, können sie ihre volle Anschlussleistung umsetzen. Beim Anschluss am 70-Volt-Abgriff nehmen sie nur noch die halbe Anschlussleistung auf. Somit kann der Verstärker an seinem 70-Volt-Ausgang die doppelte Anzahl an Lautsprechern versorgen, wobei jeder einzelne Lautsprecher nur noch die Hälfte seiner Nennleistung erhält. In der gleichen Weise nehmen die Lautsprecher am 50-Volt-Abgriff nur noch ein Viertel ihrer Leistung auf, sodass ein Verstärker in diesem Fall die vierfache Anzahl an Lautsprechern versorgen kann.

Die Übertrager in den Lautsprechern haben ebenfalls entsprechende Abgriffe, wobei dort nicht die Spannung angegeben ist, sondern die Leistungsaufnahme des Lautsprechers (z.B. P, P/2, P/4 oder 6 W, 3 W, 1,5 W). Diese Anschlüsse am Lautsprecherübertrager werden in der gleichen Weise benutzt, wie die am Ausgangsübertrager des Verstärkers, nur dass hier die Leistungsaufnahme der Lautsprecher an die verfügbare Verstärkerleistung angepasst wird.

Wenn die Leistungsaufnahme aller Lautsprecher beschränkt werden soll, ist es einfacher und effizienter, die Umschaltung am Übertrager des Verstärkers vorzunehmen. Es ist auch möglich, die Leistungsaufnahme einer bestimmten Anzahl von Lautsprechern über die Abgriffe der Lautsprecherübertrager zu verringern, während die übrigen Lautsprecher mit voller Leistung betrieben werden.

Die Nachteile der 100-Volt-Technik sind ebenfalls zu berücksichtigen. Nach einer bestimmten Kabellänge bei einem gegebenen Kabelquerschnitt treten ohmsche Verluste auf, weshalb eine maximal zulässige Kabellänge bei der Planung bedacht bzw. der Kabelquerschnitt größer dimensioniert werden muss. Durch den Einsatz von Übertragern im Signalweg treten zusätzlich gewisse Verluste auf. Wenn z.B. 10 W Lautsprecherleistung benötigt werden, führt die Verwendung eines Übertragers mit Verlusten von 1,5 dB dazu, dass der Verstärker eine Ausgangsleistung von 14,13 W bereitstellen muss. Die Impedanz der Übertrager ist darüber hinaus frequenzabhängig, was den Gesamtfrequenzgang, vor allem bei niedrigen Frequenzen ungünstig beeinflusst, weshalb auch die Lautsprecher für verschiedene Anwendungszwecke vom Hersteller gekennzeichnet werden.

\subsection{Technische Daten}
\label{sub:Technische Daten1}
Der im Datenblatt eines Verstärkers angegebene \textbf{Frequenzgang} enthält die Frequenzen an den Punkten, an denen die Kurve um einen Pegel von 3 dB abfällt. Diese beiden Frequenzen kennzeichnen die Bandbreite des Verstärkers. Bei Angaben zu Leistungsverstärkern ist dieser Pegelabfall bei 10 dB angesetzt.\\
Die \textbf{Leistungsbandbreite} ist der Frequenzbereich, in dem der Verstärker seine Nennleistung mit einer Toleranz von 3 dB abgeben kann.\\
Die \textbf{Nenn-Ausgangsleistung} ist die Leistung, die der Verstärker bei einer gegebenen Frequenz oder in einem Frequenzband an die Nenn-Abschlussimpedanz abgeben kann, ohne dass der Nennwert des Klirrfaktors überschritten wird. Die Definition ist in \textit{IEC 268-3} aufgeführt. \\
Der \textbf{Klirrfaktor} ist hierbei das Verhältnis des Effektivwerts der Oberschwingungen zum Effektivwert des Wechselanteils. D.h. Klirren ist eine nichtlineare Verzerrung, bei der sich im Ausgangssignal Frequenzen zeigen, welche im Eingangssignal nicht vorkommen, wodurch sich auch die Wellenform des Signals ändert. Bei der Übersteuerung des Eingangspegels, die \textbf{stets zu vermeiden ist}, wird z.B. der positive und negative Amplitudenbereich abgeschnitten.
Der Klirrfaktor wird als Prozentwert in \textit{THD (Total Harmonic Distortion)} angegeben. Je höher er ist, desto mehr unerwünschte Frequenzen verfälschen das Eingangssignal.

\section{Hardwareinstallation}
\label{sec:Hardwareinstallation}
Die Fehlerhafte Erdung einer Beschallungsanlage kann den Betrieb durch Brummen, Verzerrungen oder Instabilität empfindlich stören. Es kann zudem zu Überlastungen 
kommen, durch die bestimmte elektrische Bauteile ausfallen. Die häufigste Ursache solcher Probleme sind \textbf{Erdschleifen} durch schlecht geplante Verkabelungen. Sie entstehen durch den Anschluss verschiedener Komponenten der Anlage an unterschiedliche Erdanschlüsse.

Die Erdung über das Netzkabel ist die sog. Schutzerde, die mögliche gefährdende Spannungen bei elektrischen Störungen nach Erde ableitet.
Bei professionellen Beschallungsanlagen ist diese auch mit der Abschirmung der Anlage verbunden und dient als Systemerde. Systemerde darf nicht mit Störsignalen behaftet sein. Soweit möglich sollte ein getrennter Erdanschluss ins Erdreich verlegt werden, der mit dem Systemgestell verbunden ist.

Geräte für Beschallungsanlagen sind so gebaut, dass sie in Schränke mit genormter Frontbreite, die sog. \textit{19''-Gestelle}, integriert werden können. Zur Vereinfachung der Berechnung des beanspruchten Raumes im Gestell, wurde eine genormte \textit{Höheneinheit (HE)} festgelegt, die 1,75 Zoll beträgt. Die meisten Leistungsverstärker haben dann z.B. eine Höhe von 1 bzw. 2 HE. So kann einfach die Anzahl der Geräte, die in das Gestell passen, berechnet werden.

\section{Lautsprecher}
\label{sec:Lautsprecher}
Zur Planung von Beschallungsanlagen ist es entscheidend, zu wissen, welche Arten von Lautsprechern vorhanden sind und wo ihre Stärken und Schwächen liegen.

\subsection{Lautsprechertypen}
\label{sub:Lautsprechertypen}
\textbf{Konuslautsprecher} sind die am häufigsten eingesetzte Variante. Um einwandfrei zu funktionieren, müssen sie in einem passenden Gehäuse verbaut sein. Ihre hohe Breitbandigkeit macht sie besonders geeignet für Sprach- und Musikwiedergabe. Durch den geringeren Wirkungsgrad und Schalldruckpegel im Vergleich zu Druckkammerlautsprechern wird ihr Einsatzbereich in Räumen mit starkem Umgebungsgeräusch bzw. großen Abständen zu den Zuhörern beschränkt.
Eine sehr verbreitete Ausführung hiervon ist ein zylinderförmiges Gehäuse aus ABS-Kunststoff, der ideal für den Betrieb in Innenräumen und im Freien geeignet ist.  \\
Der \textbf{Kardioid-Kugellautsprecher} ist ein Lautsprecher in einem kugelförmigen Gehäuse, der spezifisch für die Sprachübertragung entwickelt wurde. Normale Konuslautsprecher haben eine mit der Frequenz zunehmende Richtwirkung. Wegen der eingebauten akustischen Filter hat der Kardioid-Kugellautsprecher eine sehr definiert gerichtete Abstrahlung im gesamten Frequenzbereich.\\
\textbf{Schallgruppen} bestehen aus mehreren Lautsprechern (gewöhnlich 4 bis 10), die direkt übereinander in einem senkrechten Gehäuse montiert sind. Sie haben in der Vertikalen, besonders bei hohen Frequenzen, eine sehr stark gebündelte Abstrahlung. Daher eignen sie sich für Räume mit starkem Nachhall, z.B. Kirchen oder Sporthallen. \\
\textbf{Kardioid-Schallgruppen} sind eine günstigere Alternative, die nach dem gleichen Prinzip, wie Kardioid-Kugellautsprecher arbeiten. Der Öffnungswinkel bleibt so auch bei niedrigen Frequenzen klein und ungefähr im gleichen Bereich wie bei den Höhen. Sie lassen sich besonders in akustisch problembehafteten Umgebungen einsetzen, da eine höhere Sicherheit für die Verständlichkeit zu erwarten ist.

\subsection{Technische Daten}
\label{sub:Technische Daten2}
\begin{compactenum}
\item Die Belastbarkeit eines Lautsprechers wird in Watt angegeben. Ein 10 W Lautsprecher muss eine maximale Verstärkerleistung von 10 W verarbeiten können.
\item Die Empfindlichkeit eines Lautsprechers wird über den Schalldruckpegel in dB (SPL) angegeben. Die Messung erfolgt bei 1 kHz, in 1 m Abstand bei einer zugeführten Leistung von 1 W.
\item Wenn die Ausgangsleistung eines Lautsprechers verdoppelt wird, nimmt der Schalldruckpegel um 3 dB zu. Wenn also die Empfindlichkeit eines Lautsprechers bekannt ist, lässt sich einfach sein Schalldruckpegel für jede beliebige Eingangsleistung berechnen. Wenn ein Lautsprecher beispielsweise eine Empfindlichkeit von 99 dB (bei 1 W und 1 m Abstand) hat, erhöht sich der Schalldruckpegel bei einer Leistung von 2 W um 3 dB auf 102 dB. 4 W bewirken eine Zunahme auf 105 dB, etc., bis die Nennleistung erreicht wird.
\item Sind zwei Lautsprecher dicht nebeneinander angeordnet und senden gleichzeitig dasselbe Signal mit gleicher Phase, kommt beim Zuhörer ein Schalldruckpegel an, der 6 dB höher ist.
\item Werden dieselben Lautsprecher weiter voneinander entfernt, treten auch bei kleinen Abständen von nur 1 m, beim Zuhörer Laufzeit- und somit Phasenverschiebungen auf, wodurch der Schalldruckpegel nur um 3 dB steigt.
\item Der Schalldruckpegel nimmt mit zunehmender Entfernung kontinuierlich ab. Mit jeder Verdopplung des Abstandes zum Lautsprecher sinkt der Schalldruckpegel um 6 dB.
\item Diese Beispiele beschränken sich auf Lautsprecher, dessen Empfindlichkeitsmessung exakt in der Mitte der Lautsprecherachse und mit einem Ton von 1 kHz durchgeführt wurde. Ein Richtdiagramm zeigt stattdessen, wie der Schalldruck von der Frequenz und von dem Winkel zur Lautsprecherachse (0°) abhängt. Der Winkelbereich zwischen den beiden Punkten, an denen der Schalldruckpegel 6 dB ist, ist der Öffnungswinkel, der normalerweise bei 4 kHz definiert ist.
\end{compactenum}

\begin{figure}[H]
\centering
\includegraphics[width=0.8\linewidth]{images/Richtwirkung}
\caption{Richtdiagramm eines Konuslautsprechers abhängig von der Frequenz}
\label{fig:richtdiagramm}
\end{figure}

\section{Akustische Umgebung}
\label{sec:Akustische Umgebung}
Klangcharakter und Ausbreitungseigenschaften des Schalls werden durch die Umgebung grundlegend beeinflusst. Das gleiche Audiosignal klingt in einem Stadion völlig anders als in einer großen, halligen Kirche oder in einem stark bedämpften Lesesaal. Man unterscheidet im Allgemeinen zwischen zwei Situationen: in Innenräumen und im Freien.

In beiden Fällen sind die Hauptziele:
\begin{compactenum}
\item Sprachverständlichkeit: die deutliche Übertragung einer Botschaft für die Zuhörer
\item Wiedergabequalität: die möglichst unverfälschte Übertragung von z.B. Musik 
\end{compactenum}

\subsection{Beschallung im Freien}
\label{sub:Beschallung im Freien}
Im Freien müssen unterschiedliche Faktoren, die die Ausbreitung von Schall beeinflussen, berücksichtigt werden.

Für die Berechnung des Versorgungsbereiches ist Kenntnis vom \textbf{Öffnungswinkel} eines spezifischen Lautsprechers notwendig. Je nach Umgebung und besonderen Anforderungen kann es erforderlich sein, Lautsprecher mit einem großen Öffnungswinkel, die den Schall über eine große Fläche verbreiten, einzusetzen. In anderen Anwendungen darf der Schall nur sehr definiert in bestimmte Richtungen gestrahlt werden. Dies ist besonders dort wichtig, wo eine unnötig breite Abstrahlung des Schalls nicht nur eine Verschwendung von Verstärkerleistung bedeuten würde, sondern der Schall auch von anderen Gebäuden reflektiert werden oder Personen in anderen Gebäuden stören könnte. Eine unkontrollierte Schallausbreitung kann dazu führen, dass zu viel Lautsprechersignal durch das Mikrofon wieder aufgenommen wird, wodurch akustische Rückkopplungen entstehen.

Wenn Schall im Freien wiedergegeben wird, wo keine Objekte vorhanden sind, die \textbf{Reflexionen} verursachen, hört der Zuhörer ausschließlich den Direktschall. Der Schalldruck nimmt mit jeder Verdopplung der Entfernung zur Schallquelle um 6 dB ab.

Wenn ein Lautsprecher eine Empfindlichkeit von 100 dB hat, kann der Schalldruckpegel in dB (SPL) bei 26 m Entfernung und einer Eingangsleistung von 10 W wie folgt berechnet werden:\\[1ex]
Dämpfung: $20\cdot\log26 = 28,3\,dB$ \\
Zunahme des Schalldrucks: $10\cdot\log10 = 10\,dB$ \\
Gesamtergebnis: $100 - 28,3 + 10 = 81,7\,dB$ \\
Gesamtformel: $L_{dir} = L_s + 10\cdot\log P_{el} - L_Q - 20\cdot\log r$

\textbf{Brechung} oder \textbf{Beugung} des Schalls tritt dann auf, wenn der Schall von einem Medium in ein anderes übergeht. Dieser Effekt zeigt sich auch dann, wenn der Schall Luftschichten mit verschiedenen Temperaturen und unterschiedlichen Ausbreitungsgeschwindigkeiten durchläuft.

Obwohl \textbf{Reflexionen} eher in Innenräumen problematisch sind, treten auch im Freien an Gebäuden Reflexionen auf, die sehr störende Echos verursachen. Wenn der Laufzeitunterschied zwischen Direktschall und reflektierendem Schall mehr als 50 ms beträgt, nimmt der Hörer diese Reflexion wahr und hört sie als Echo des ursprünglichen Signals. Ausgehend von einer Schallgeschwindigkeit von 340 m/s in der Luft, entspricht ein Laufzeitunterschied von 50 ms einer Entfernung von 17 m. Wenn der Unterschied zwischen der direkten Entfernung und der Entfernung über die reflektierende Fläche kürzer als 17 m ist, verstärkt bzw. unterstützt der reflektierte Schall den Direktschall statt ein Echo zu verursachen.

Der Qualitätseindruck einer Lautsprecher- bzw. Beschallungsanlage kann durch \textbf{Umgebungsgeräusche} deutlich beeinflusst werden. Das konstante Rauschen vorbeifahrender Fahrzeuge, der Lärm einer großen Industrieanlage usw. kann signifikante Pegel erreichen, die berücksichtigt werden müssen. Wenn der Schallpegel einer Quelle in einer Situation gemessen wird, in der Umgebungsgeräusche auftreten, muss der Pegel des Umgebungsgeräuschs vom gemessenen Gesamtwert abgezogen werden, um den tatsächlichen Schallpegel der Quelle zu erhalten.

\subsection{Beschallung in Innenräumen}
\label{sub:in Innenraeumen}
Bei der Planung von Beschallungsanlagen für Innenräume kann eine Reihe von Problemen auftreten. Da sich die Zuhörer i.d.R. in einiger Entfernung zur Schallquelle befinden, werden hohe Frequenzen teilweise durch die Luft absorbiert, während bei niedrigen Frequenzen Probleme mit \textbf{Nachhall durch Reflexionen} an Wänden und Decke auftreten. D.h. in einer halligen Umgebung und mit zunehmender Entfernung zum Zuhörer treten zwei Hauptprobleme auf:
\begin{compactenum}
\item eine frequenzmäßige Beschneidung des ursprünglichen Sprachspektrums
\item Hall und Reflexionen im tieffrequenten Bereich des Sprachspektrums
\end{compactenum}
Hieraus ergibt sich, dass die Zuhörer zwar eine ausreichende Lautstärke erhalten, die Konsonanten in der Sprache durch den Nachhall jedoch bedämpft und verdeckt werden, wodurch die Verständlichkeit leidet.

Wenn eine Schallquelle in einem Raum von Wänden und Decke umgeben ist, reflektieren und absorbieren sie den Schall jeweils zu bestimmten Teilen. Die Intensität des reflektierten Schalls $I_{ref}$ ist geringer als die des auftreffenden Schalls $I_{inc}$. Ein Anteil $\alpha$ (Absorptionskoeffizient) des auftreffenden Schalls geht somit verloren.\\ [1em]
$I_{ref} = (1-\alpha)\cdot I_{inc}$ mit $\alpha + r = 1$

Wenn der Schall vollständig reflektiert wird (r = 1), wird durch das Material kein Schall absorbiert ($\alpha$ = 0).

Bei der Beschallung in einem Raum erreicht ein Teil des Schalls die Zuhörer auf direktem Wege, der wesentlichste Teil jedoch gelangt über eine oder mehrere aufeinanderfolgende Reflexionen zu den Zuhörern. Dieser resultierende Effekt wird als \textbf{Nachhall} bezeichnet und führt zum Aufbau eines diffusen Schallfelds im gesamten Raum, dem \textit{Nachhallfeld}. Die Stärke dieses Feldes wird durch die Art der Schallquelle, die Größe des Raums und die Nachhallzeit bestimmt.

Die \textbf{Nachhallzeit} (T bzw. $RT_{60}$) eines Raums ist als Zeit definiert, nach welcher der Schallpegel des Nachhallfelds um 60 dB abgenommen hat. Dabei wird vorausgesetzt:
\begin{compactenum}
\item Die Nachhallzeit ist, unabhängig von der Position der Schallquelle oder des Zuhörers, im gesamten Raum gleich.
\item Zu geringe Verständlichkeit in einem Raum ist fast immer Folge einer zu langen Nachhallzeit.
\item I.d.R. hat die Raumform wenig Einfluss auf die Nachhallzeit.
\item Sie wird durch das Raumvolumen und das Maß der Schallabsorption bestimmt.
\end{compactenum}
Die Nachhallzeit wird durch folgende Formel bestimmt: $RT_{60} = 0,161\cdot \frac{Volumen}{Absorption}$ \\
Absorption: $A = S + 4\cdot m\cdot V + n\cdot A_p$ und $S = \Sigma(S_i\alpha_i)$ \\
Somit gilt: $RT_{60} = T = 0,161\cdot \frac{V}{S + 4\cdot m\cdot V + n\cdot A_p}$ \\ [1em]

Ein umfassendes Verständnis der verschiedenen Schallfelder im Raum ist besonders wichtig. Nur der Direktschall sorgt für die Verständlichkeit, alles andere beeinträchtigt diese. Der Direktschall wird durch das Ohr als Summe aller sprachbezogener Schallereignisse in einem Zeitabschnitt von 0 - 35 ms wahrgenommen. Dies sind der direkte Schall, der unmittelbar von der Schallquelle ankommt und der indirekte Schall infolge von Reflexionen, solange dieser innerhalb des angegebenen Zeitfensters bleibt. \\ [1em]
Direktschall: $L_{dir}=L_s + 10\cdot\log P_{el} - L_{Q1} - 20\cdot\log r_1$ \\
Indir. Schall: $L_{indir}=L_s + 10\cdot\log P_{el} - L_{Q2} + 10\cdot\log(1 - \alpha_1)$

